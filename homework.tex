\documentclass[UTF8,zihao=-4]{ctexart}
\usepackage{graphicx}
\usepackage{geometry}
\usepackage{siunitx}
\usepackage{amsmath}
\usepackage{amsfonts}
\usepackage{amssymb}
\usepackage{pifont}
\usepackage{listings}
\lstset{
	breaklines,
	tabsize=4,
	basicstyle=\ttfamily
}
\geometry{a4paper,centering,scale=0.8}
\setCJKfamilyfont{namekai}{AR PL UKai CN}
\title{\heiti 计算机体系结构\quad 第二次作业}
\author{\CJKfamily{namekai} PB17000005\quad 赵作竑}
\date{\kaishu \today}
\begin{document}
	\maketitle
	\begin{enumerate}
		\item[1.]
		\begin{enumerate}
			\item[a.]书上说对寄存器的访问是``在前半个周期写,在后半个周期读'',这样第一条指令和第四条指令之间就没有数据相关了(第一条指令在第五段写回,是在前半个周期,第四条指令译码访问寄存器是在后半个周期,读取到的是写回的新值)。这道题我就按照书上的方案去做了。

			从指令1到指令2存在对寄存器R1的RAW相关和WAW相关;

			从指令1到指令3存在对寄存器R1的WAW相关;

			从指令2到指令3存在对寄存器R1的WAW相关;

			从指令4到指令5存在对寄存器R2的RAW相关;

			从指令5到指令6存在对寄存器R4的RAW相关。

			\item[b.]根据题意,寄存器的读取与写入通过寄存器堆进行转发,所以如果两条指令有数据相关,只需要停顿两个周期。这里需要处理的只有RAW相关,从指令1取指阶段开始计,两个RAW相关分别在第2个周期和第7个周期。最后的BNEZ指令,按照书上的说法,可以把判断提到ID阶段完成,不需要用ALU,所以只需要进行一次无用的取指,在下一个周期就可以得到正确的PC值。接下来看这个循环:每次执行循环,R2都会增加4,那么这个循环总共执行$396\div 4=99$次之后,R4为零,循环结束。从指令1开始取指到指令1下次取指之前,经过了13个时钟周期。最后一次,还要花费额外的三个周期,已完成指令6。所以这一循环的执行需要$99\times 13+3=1290$个周期。

			\begin{center}
				\begin{tabular}{|c|c|c|c|c|c|c|c|}
					\hline
					No. &  1  &   2   &  3  &  4  &   5   &   6   & 7  \\ \hline
					 1  & IF  &       &     &     &       &       &    \\
					 2  & ID  & Stall &     &     &       &       &    \\
					 3  & EX  & Stall &     &     &       &       &    \\
					 4  & MEM &  IF   &     &     &       &       &    \\
					 5  & WB  &  ID   & IF  &     &       &       &    \\
					 6  &     &  EX   & ID  & IF  &       &       &    \\
					 7  &     &  MEM  & EX  & ID  & Stall &       &    \\
					 8  &     &  WB   & MEM & EX  & Stall &       &    \\
					 9  &     &       & WB  & MEM &  IF   &       &    \\
					10  &     &       &     & WB  &  ID   & Stall &    \\
					11  &     &       &     &     &  EX   & Stall &    \\
					12  &     &       &     &     &  MEM  &  IF   &    \\
					13  &     &       &     &     &  WB   &  ID   & IF \\ \hline
					14  & IF  &       &     &     &       &  EX   &    \\
					15  & ID  & Stall &     &     &       &  MEM  &    \\
					16  & EX  & Stall &     &     &       &  WB   &    \\ \hline
				\end{tabular}
			\end{center}
			\item[c.]如果五级流水线拥有完整旁路定向路径,那么处理数据相关时,无需停顿流水线,另外处理分支是采用的是预测转移失败策略,也就是说先对指令7取指。下面的时序图中是分支转移成功的情况,会浪费一个周期。循环进行99次,花费$99\times 7=693$个周期,最后一次还要额外花费3个周期完成指令6,所以总共是$693+3=696$个周期。

			\begin{center}
				\begin{tabular}{|c|c|c|c|c|c|c|c|}
					\hline
					No. &  1  &  2  &  3  &  4  &  5  &  6  &  7   \\ \hline
					 1  & IF  &     &     &     &     &     &      \\
					 2  & ID  & IF  &     &     &     &     &      \\
					 3  & EX  & ID  & IF  &     &     &     &      \\
					 4  & MEM & EX  & ID  & IF  &     &     &      \\
					 5  & WB  & MEM & EX  & ID  & IF  &     &      \\
					 6  &     & WB  & MEM & EX  & ID  & IF  &      \\
					 7  &     &     & WB  & MEM & EX  & ID  &  IF  \\ \hline
					 8  & IF  &     &     & WB  & MEM & EX  & idle \\
					 9  & ID  &     &     &     & WB  & MEM & idle \\
					10  & EX  &     &     &     &     & WB  & idle \\ \hline
				\end{tabular}
			\end{center}
			\pagebreak
			\item[d.]如果采用预测转移成功策略进行分支处理,那么在循环进行过程中的时序如图所示,进行99次循环耗费$99\times 6=594$个周期,最后一次还需要额外的4个周期完成指令6,所以这种情况下一共需要$594+4=598$个周期。

			\begin{center}
				\begin{tabular}{|c|c|c|c|c|c|c|}
					\hline
					No. &  1  &  2  &  3  &  4  &  5  &  6  \\ \hline
					 1  & IF  &     &     &     &     &     \\
					 2  & ID  & IF  &     &     &     &     \\
					 3  & EX  & ID  & IF  &     &     &     \\
					 4  & MEM & EX  & ID  & IF  &     &     \\
					 5  & WB  & MEM & EX  & ID  & IF  &     \\
					 6  &     & WB  & MEM & EX  & ID  & IF  \\ \hline
					 7  & IF  &     & WB  & MEM & EX  & ID  \\
					 8  & ID  &     &     & WB  & MEM & EX  \\
					 9  & EX  &     &     &     & WB  & MEM \\
					10  & MEM &     &     &     &     & WB  \\ \hline
				\end{tabular}
			\end{center}
		\end{enumerate}
		\item[2.]假设计算
		\begin{equation*}
			\prod_{i=1}^4\left( A_i+B_i\right)
		\end{equation*}
		使用的汇编指令是这样的:
		\begin{lstlisting}
1. ADD A1, A1, B1
2. ADD A2, A2, B2
3. ADD A3, A3, B3
4. ADD A4, A4, B4
5. MUL A1, A1, A2
6. MUL A1, A1, A3
7. MUL A1, A1, A4
		\end{lstlisting}
		我们首先画出它的时空图。可以看到,执行7条指令总共花费了$12\Delta t$的时间,因此吞吐率为:
		\begin{equation*}
			\text{Throughput}=\frac{7}{12\Delta t}
		\end{equation*}
		通过流水线执行时间与非流水线执行所花时间相除,得到加速比为:
		\begin{equation*}
			\text{Speedup}=\frac{7\times 6\Delta t}{4\times (\Delta t+2\Delta t+\Delta t+\Delta t)+3\times (\Delta t+\Delta t+\Delta t)}=\frac{42}{29}\approx 1.45
		\end{equation*}
		最后是效率的计算。首先,在$12\Delta t$的时间里,总的时空区为
		\begin{equation*}
			\text{All}=12\Delta t\times 7=84\Delta t
		\end{equation*}
		使用的时空区为
		\begin{align*}
			\text{Used}&=\sum_{i=1}^{5}\text{Used}_i \\
			&= 7\times \Delta t+3\times \Delta t+4\times 2\Delta t+4\times \Delta t+7\times \Delta t \\
			&= 29\Delta t
		\end{align*}
		将两者相除,得到流水线的效率为
		\begin{align*}
			\text{Efficiency}=\frac{\text{Used}}{\text{All}}=\frac{29\Delta t}{84\Delta t}=\frac{29}{84}\approx 34.5\%
		\end{align*}
		\begin{center}
			\begin{tabular}{|c|c|c|c|c|c|c|c|}
				\hline
				Time & 1 &   2   &   3   &   4   &   5   &   6   &   7   \\ \hline
				 1   & 1 &       &       &       &       &       &       \\
				 2   & 3 &   1   &       &       &       &       &       \\
				 3   & 3 & Stall &   1   &       &       &       &       \\
				 4   & 4 &   3   & Stall &   1   &       &       &       \\
				 5   & 5 &   3   & Stall & Stall &   1   &       &       \\
				 6   &   &   4   &   3   & Stall &   2   &   1   &       \\
				 7   &   &   5   &   3   & Stall & Stall &   2   &   1   \\
				 8   &   &       &   4   &   3   &   5   & Stall &   2   \\
				 9   &   &       &   5   &   3   &       & Stall & Stall \\
				 10  &   &       &       &   4   &       &   5   & Stall \\
				 11  &   &       &       &   5   &       &       & Stall \\
				 12  &   &       &       &       &       &       &   5   \\ \hline
			\end{tabular}
		\end{center}
		\item[3.]
		\begin{enumerate}
			\item[a.]如果只考虑数据相关,那么假设两台机器都执行$n$条指令,那么第一台机器有$n/5$次Stall,第二台机器有$3n/8$次Stall。忽略排空与填充流水线的时间,两台机器总共经过的周期分别为$n+n/5=6n/5$与$n+3n/8=11n/8$,分别乘以两台机器的时钟周期,得到两台机器的执行时间分别为:
			\begin{align*}
				T_1&=\SI{1}{ns}\times \frac{6n}{5}=n\times \SI{1.2}{ns} \\
				T_2&=\SI{0.6}{ns}\times \frac{11n}{8}=n\times \SI{0.825}{ns}
			\end{align*}
			由此可以算出加速比:
			\begin{equation*}
				\text{Speedup}=\frac{T_1}{T_2}=\frac{n\times \SI{1.2}{ns}}{n\times \SI{0.825}{ns}}\approx 1.45
			\end{equation*}
			\item[b.] 首先考虑第一种机器。计算每条指令由于数据相关而导致的停顿周期数,题目中说道每5条指令停顿一个周期,因此平均每条指令停顿0.2个周期。平均每条指令因为分支预测错误而停顿的周期数为$20\% \times 5\% \times 2=0.02$。所以平均每条指令停顿的周期为$0.2+0.02=0.22$,这条流水线的CPI为:
			\begin{equation*}
				\text{CPI}_1=1+0.22=1.22
			\end{equation*}
			接下来考虑第二种机器。还是先计算每条指令由于数据相关而导致的停顿周期数:$3/8=0.375$。之后是平均每条指令由于预测错误而停顿的周期数:$20\% \times 5\% \times 5=0.05$,由此得到第二条流水线的CPI为:
			\begin{equation*}
				\text{CPI}_2=1+0.375+0.05=1.425
			\end{equation*}
		\end{enumerate}
	\end{enumerate}
\end{document}